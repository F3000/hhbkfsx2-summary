\section{Deutsch und Kommunikation}


%%% Anfang: tl;dr
\subsection{tl;dr - Zusammenfassung der Zusammenfassung}
%%% Ende: tl;dr

%%% Anfang: Lernen
\subsection{Lernen}


%%% Anfang: Lernen > Physiologie
\subsubsection{Physiologische Voraussetzungen des Lernerfolges}

%%% Anfang: Lernen > Typen
\subsubsection{Welche Lerntypen gibt es?}

\paragraph{Visueller Lerntyp}~\\
\paragraph{Haptischer Lerntyp}~\\
\paragraph{Auditiver Lerntyp}~\\
\paragraph{Kommunikativer Lerntyp}~\\

%%% Anfang: Lernen > Methoden
\subsubsection{Lernmethoden}

\paragraph{10 Lernmethoden}~\\
\begin{itemize}
	\item Notizen
	\item Markieren
	\item Mindmap
	\item Case Studies
	\item Karteikarten
\end{itemize}

%%% Anfang: Lernen > Faktoren
\subsubsection{Äußere Einflussfaktoren auf den Lernerfolg}
\paragraph{Einfluss der direkten Umgebung}~\\
\paragraph{Einfluss des sozialen Umfeldes}~\\
\paragraph{Einfluss der Ernährung}~\\
Das Gehirn kann im Gegensatz zum Rest des Körpers nur mit Glukose umgehen. Generell gilt, je höher der Glukosespiegel ist, desto besser können wir uns konzentrieren und desto besser ist unsere geistige Leistungsfähigkeit. Jedoch gibt es ein oberes Limit, ab dem der Körper vermehrt Insulin produziert, welches den Glukosespiegel rapide sinken lässt. Es folgen Müdigkeit, Konzentrationsstörungen und generell eine geringere Leistungsfähigkeit. Statt stark zuckerhaltige Lebensmittel zu verzehren, sollten also Lebensmitteln mit einem hohen Anteil an komplexen Kohlenhydraten bevorzugt werden, damit der Glukosespiegel nicht zu schnell steigt und über einen längeren Zeitraum konstant bleibt.
Konzentrationsschwäche kann aber auch dann auftreten, wenn dem Körper bestimmte Mineralien fehlen wie bspw. Eisen. Für optimale Leistungsfähigkeit sollte auf eine gesunde Ernährung geachtet werden.
\paragraph{Einfluss von Drogen}~\\

%%% Ende: Lernen
%%%%%%%%%%%%%%%%%%%%%%%%%%%%%%%%%%%%%%%%%%%%%%%%%%%%%%%%%%%%%%%%%%%%%%%%%%%%%%%%